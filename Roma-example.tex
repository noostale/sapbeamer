\documentclass{beamer} % Beamer class for presentations

% =============================================================================
% Language and Encoding Configuration
% =============================================================================

\usepackage[italian]{babel} % Italian language support

% Uncomment based on your compiler
% For pdfLaTeX:
\usepackage[utf8]{inputenc}   % UTF-8 encoding
\usepackage[T1]{fontenc}      % Font encoding
\usepackage{lmodern}          % Latin Modern font

\usepackage{ragged2e} % Add this in the preamble for justifying text


% For LuaLaTeX or XeLaTeX:
%\usepackage{fontspec}        % Font selection and Unicode support


% =============================================================================
% Presentation Metadata
% =============================================================================

\title{I numeri primi sono infiniti}
\author[Euclide]{Euclide di Roma\\\texttt{euclide@roma.edu}}
\date[\today]{\today}
\institute[Roma]{DIPARTIMENTO DI INGEGNERIA INFORMATICA\\[-0.1cm] AUTOMATICA E GESTIONALE ANTONIO RUBERTI}

%insert justified institute using \justifying


% =============================================================================
% Theme Configuration
% =============================================================================

\usetheme[
    pageofpages=/,      % String between current and total frames
    titleline=true,      % Enable line below frame titles
]{Roma}

% =============================================================================
% Theorem Styles
% =============================================================================

\theoremstyle{definition}
\newtheorem{definizione}{Definizione}

\theoremstyle{plain}
\newtheorem{teorema}{Teorema}

% =============================================================================
% Begin Document
% =============================================================================

\begin{document}

% =============================================================================
% Title Page
% =============================================================================

\titlepageframe % Create title page frame

% =============================================================================
% Table of Contents
% =============================================================================

\begin{frame}
  \frametitle{Piano della presentazione} % "Presentation Plan"
  \tableofcontents % Display table of contents
\end{frame}

% =============================================================================
% Automatic Outline at Section Start
% =============================================================================

\AtBeginSection[]{
  \begin{frame}
    \frametitle{Outline}
    \tableofcontents[currentsection] % Highlight current section
  \end{frame}
}

% =============================================================================
% Section: Introduzione
% =============================================================================

\section{Introduzione}

\begin{frame}
  \frametitle{Che cosa sono i numeri primi?} % "What are prime numbers?"
  \begin{definizione}
    Un \alert{numero primo} è un intero $>1$ che ha esattamente due divisori positivi.
  \end{definizione}
\end{frame}

% =============================================================================
% Section: L'infinità dei primi
% =============================================================================

\section{L'infinità dei primi}
\subsection{La dimostrazione}

\begin{frame}
  \frametitle{I numeri primi sono infiniti}
  \framesubtitle{Ne diamo una dimostrazione diretta}
  
  \begin{teorema}
    Non esiste un primo maggiore di tutti gli altri.
  \end{teorema}
  
  \pause % Pause before showing proof steps
  
  \begin{proof}
    \begin{enumerate}[<+->]
      \item Sia dato un elenco di primi.
      \item Sia $q$ il loro prodotto.
      \item Allora $q+1$ è divisibile per un primo $p$ che non compare nell’elenco. \qedhere
    \end{enumerate}
  \end{proof}
\end{frame}

% =============================================================================
% Section: Problemi aperti
% =============================================================================

\section{Problemi aperti}

\begin{frame}
  \frametitle{Che cosa c’è ancora da fare?}
  
  \begin{block}{Problemi risolti}
    Quanti sono i numeri primi?
  \end{block}
  
  \begin{alertblock}{Problemi aperti}
    Un \alert{numero} pari $>2$ è sempre la somma di due primi?
  \end{alertblock}
\end{frame}

% =============================================================================
% End Document
% =============================================================================

\end{document}
